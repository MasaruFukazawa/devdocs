%% Generated by Sphinx.
\def\sphinxdocclass{report}
\documentclass[letterpaper,10pt,dvipdfmx]{sphinxmanual}
\ifdefined\pdfpxdimen
   \let\sphinxpxdimen\pdfpxdimen\else\newdimen\sphinxpxdimen
\fi \sphinxpxdimen=.75bp\relax

\PassOptionsToPackage{warn}{textcomp}


\usepackage{cmap}
\usepackage[T1]{fontenc}
\usepackage{amsmath,amssymb,amstext}



\usepackage{times}


\usepackage{sphinx}

\fvset{fontsize=\small}
\usepackage[dvipdfm]{geometry}

% Include hyperref last.
\usepackage{hyperref}
% Fix anchor placement for figures with captions.
\usepackage{hypcap}% it must be loaded after hyperref.
% Set up styles of URL: it should be placed after hyperref.
\urlstyle{same}
\renewcommand{\contentsname}{Contents:}

\usepackage{sphinxmessages}
\setcounter{tocdepth}{0}



\title{画面仕様書サンプル}
\date{2020年01月10日}
\release{}
\author{Masaru Fukazawa}
\newcommand{\sphinxlogo}{\vbox{}}
\renewcommand{\releasename}{}
\makeindex
\begin{document}

\pagestyle{empty}
\sphinxmaketitle
\pagestyle{plain}
\sphinxtableofcontents
\pagestyle{normal}
\phantomsection\label{\detokenize{index::doc}}



\chapter{画面一覧}
\label{\detokenize{screen_list:id1}}\label{\detokenize{screen_list::doc}}

\begin{savenotes}\sphinxattablestart
\centering
\begin{tabulary}{\linewidth}[t]{|T|T|T|T|}
\hline
\sphinxstyletheadfamily 
分類
&\sphinxstyletheadfamily 
画面名
&\sphinxstyletheadfamily 
URL
&\sphinxstyletheadfamily 
説明
\\
\hline
分類001
&
{\hyperref[\detokenize{page_001::doc}]{\sphinxcrossref{\DUrole{doc}{Page1}}}}
&
/page/001
&
ここに画面に説明を入力する
\\
\hline
\end{tabulary}
\par
\sphinxattableend\end{savenotes}


\chapter{画面遷移図}
\label{\detokenize{screen_transition_diagram:id1}}\label{\detokenize{screen_transition_diagram::doc}}

\chapter{共通:ヘッダー}
\label{\detokenize{header:id1}}\label{\detokenize{header::doc}}

\section{説明}
\label{\detokenize{header:id2}}

\section{画面}
\label{\detokenize{header:id3}}\begin{itemize}
\item {} 
画像を挿入

\end{itemize}


\section{URL}
\label{\detokenize{header:url}}

\section{I/0定義}
\label{\detokenize{header:i-0}}
\sphinxstylestrong{入力}


\begin{savenotes}\sphinxattablestart
\centering
\begin{tabulary}{\linewidth}[t]{|T|T|T|}
\hline
\sphinxstyletheadfamily 
項目名
&\sphinxstyletheadfamily 
フィールド名
&\sphinxstyletheadfamily 
型
\\
\hline
名前
&
name
&
文字列
\\
\hline
\end{tabulary}
\par
\sphinxattableend\end{savenotes}

\sphinxstylestrong{出力}


\begin{savenotes}\sphinxattablestart
\centering
\begin{tabulary}{\linewidth}[t]{|T|T|T|}
\hline
\sphinxstyletheadfamily 
項目名
&\sphinxstyletheadfamily 
フィールド名
&\sphinxstyletheadfamily 
型
\\
\hline
名前
&
name
&
文字列
\\
\hline
\end{tabulary}
\par
\sphinxattableend\end{savenotes}


\subsection{エラー定義}
\label{\detokenize{header:id4}}

\begin{savenotes}\sphinxattablestart
\centering
\begin{tabulary}{\linewidth}[t]{|T|T|T|}
\hline
\sphinxstyletheadfamily 
項目名
&\sphinxstyletheadfamily 
エラー種別
&\sphinxstyletheadfamily 
エラーメッセージ
\\
\hline
名前
&
未入力
&
未入力です
\\
\hline&
文字数制限
&
100文字以内です
\\
\hline
\end{tabulary}
\par
\sphinxattableend\end{savenotes}


\subsection{動作仕様}
\label{\detokenize{header:id5}}\begin{itemize}
\item {} 
〇〇ボタンについて
\begin{itemize}
\item {} 
どのように挙動する

\end{itemize}

\item {} 
△△ボタンについて
\begin{itemize}
\item {} 
どのように挙動する

\end{itemize}

\end{itemize}


\chapter{共通:フッター}
\label{\detokenize{footer:id1}}\label{\detokenize{footer::doc}}

\section{説明}
\label{\detokenize{footer:id2}}

\section{画面}
\label{\detokenize{footer:id3}}\begin{itemize}
\item {} 
画像を挿入

\end{itemize}


\section{URL}
\label{\detokenize{footer:url}}

\section{I/0定義}
\label{\detokenize{footer:i-0}}
\sphinxstylestrong{入力}


\begin{savenotes}\sphinxattablestart
\centering
\begin{tabulary}{\linewidth}[t]{|T|T|T|}
\hline
\sphinxstyletheadfamily 
項目名
&\sphinxstyletheadfamily 
フィールド名
&\sphinxstyletheadfamily 
型
\\
\hline
名前
&
name
&
文字列
\\
\hline
\end{tabulary}
\par
\sphinxattableend\end{savenotes}

\sphinxstylestrong{出力}


\begin{savenotes}\sphinxattablestart
\centering
\begin{tabulary}{\linewidth}[t]{|T|T|T|}
\hline
\sphinxstyletheadfamily 
項目名
&\sphinxstyletheadfamily 
フィールド名
&\sphinxstyletheadfamily 
型
\\
\hline
名前
&
name
&
文字列
\\
\hline
\end{tabulary}
\par
\sphinxattableend\end{savenotes}


\subsection{エラー定義}
\label{\detokenize{footer:id4}}

\begin{savenotes}\sphinxattablestart
\centering
\begin{tabulary}{\linewidth}[t]{|T|T|T|}
\hline
\sphinxstyletheadfamily 
項目名
&\sphinxstyletheadfamily 
エラー種別
&\sphinxstyletheadfamily 
エラーメッセージ
\\
\hline
名前
&
未入力
&
未入力です
\\
\hline&
文字数制限
&
100文字以内です
\\
\hline
\end{tabulary}
\par
\sphinxattableend\end{savenotes}


\subsection{動作仕様}
\label{\detokenize{footer:id5}}\begin{itemize}
\item {} 
〇〇ボタンについて
\begin{itemize}
\item {} 
どのように挙動する

\end{itemize}

\item {} 
△△ボタンについて
\begin{itemize}
\item {} 
どのように挙動する

\end{itemize}

\end{itemize}


\chapter{Page1}
\label{\detokenize{page_001:page1}}\label{\detokenize{page_001::doc}}

\section{説明}
\label{\detokenize{page_001:id1}}

\section{画面}
\label{\detokenize{page_001:id2}}\begin{itemize}
\item {} 
画像を挿入

\end{itemize}


\section{URL}
\label{\detokenize{page_001:url}}\begin{itemize}
\item {} 
/page/001

\end{itemize}


\section{I/0定義}
\label{\detokenize{page_001:i-0}}
\sphinxstylestrong{入力}


\begin{savenotes}\sphinxattablestart
\centering
\begin{tabulary}{\linewidth}[t]{|T|T|T|}
\hline
\sphinxstyletheadfamily 
項目名
&\sphinxstyletheadfamily 
フィールド名
&\sphinxstyletheadfamily 
型
\\
\hline
名前
&
name
&
文字列
\\
\hline
\end{tabulary}
\par
\sphinxattableend\end{savenotes}

\sphinxstylestrong{出力}


\begin{savenotes}\sphinxattablestart
\centering
\begin{tabulary}{\linewidth}[t]{|T|T|T|}
\hline
\sphinxstyletheadfamily 
項目名
&\sphinxstyletheadfamily 
フィールド名
&\sphinxstyletheadfamily 
型
\\
\hline
名前
&
name
&
文字列
\\
\hline
\end{tabulary}
\par
\sphinxattableend\end{savenotes}


\subsection{エラー定義}
\label{\detokenize{page_001:id3}}

\begin{savenotes}\sphinxattablestart
\centering
\begin{tabulary}{\linewidth}[t]{|T|T|T|}
\hline
\sphinxstyletheadfamily 
項目名
&\sphinxstyletheadfamily 
エラー種別
&\sphinxstyletheadfamily 
エラーメッセージ
\\
\hline
名前
&
未入力
&
未入力です
\\
\hline&
文字数制限
&
100文字以内です
\\
\hline
\end{tabulary}
\par
\sphinxattableend\end{savenotes}


\subsection{処理}
\label{\detokenize{page_001:id4}}

\subsubsection{API呼び出し}
\label{\detokenize{page_001:api}}\begin{itemize}
\item {} 
〇〇APIを呼び出す
\begin{itemize}
\item {} 
〇〇を取得するため

\end{itemize}

\end{itemize}


\subsubsection{ボタン挙動}
\label{\detokenize{page_001:id5}}\begin{itemize}
\item {} 
〇〇ボタンについて
\begin{itemize}
\item {} 
どのように挙動する

\end{itemize}

\item {} 
△△ボタンについて
\begin{itemize}
\item {} 
どのように挙動する

\end{itemize}

\end{itemize}


\chapter{Indices and tables}
\label{\detokenize{index:indices-and-tables}}\begin{itemize}
\item {} 
\DUrole{xref,std,std-ref}{genindex}

\item {} 
\DUrole{xref,std,std-ref}{modindex}

\item {} 
\DUrole{xref,std,std-ref}{search}

\end{itemize}



\renewcommand{\indexname}{索引}
\printindex
\end{document}